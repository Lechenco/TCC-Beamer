\documentclass[10pt]{beamer}

\usetheme[progressbar=frametitle]{metropolis}
\usepackage{appendixnumberbeamer}

\usepackage{booktabs}
\usepackage[scale=2]{ccicons}

\usepackage{pgfplots}
\usepgfplotslibrary{dateplot}

\usepackage{xspace}
\newcommand{\themename}{\textbf{\textsc{metropolis}}\xspace}

\title{Detecção de artefatos de arritmia utilizando Máquinas de Vetores de Suporte e Coeficientes de Energia Wavelet}
\subtitle{Proposta de TCC}
% \date{\today}
\date{2020}
\author{Gabriel Lechenco Vargas Pereira \\
Cristiano Marcos Agulhari}
\institute{Universidade Tecnológica Federal do Paraná - UTFPR}
% \titlegraphic{\hfill\includegraphics[height=1.5cm]{logo.pdf}} 

\begin{document}
\setbeamertemplate{frame footer}{Universidade Tecnológica Federal do Paraná}

\maketitle

\begin{frame}{Sumário}
  \setbeamertemplate{section in toc}[sections numbered]
  \tableofcontents[hideallsubsections]
\end{frame}

\section{Introdução}

\begin{frame}{Introdução}
    Uma rede pode ser dividida nos seguintes planos:
    \begin{itemize}
        \alert<2>{\item Plano de Dados}
        \alert<2>{\item Plano de Controle}
        \item Plano de Gerenciamento
    \end{itemize}
\end{frame}

\section{Fundamentação Teórica}

\begin{frame}{Eletrocardiograma}
    
\end{frame}

\begin{frame}{Máquinas de Vetores de Suporte (SVM)}
    
\end{frame}

\begin{frame}{Wavelet}
    
\end{frame}

\section{Trabalhos Relacionados}

\section{Proposta}

\begin{frame}{Proposta}
    
\end{frame}

\begin{frame}{Pré-processamento}
    
\end{frame}

\begin{frame}{Aprendizado de Máquina}
    
\end{frame}

\begin{frame}{Testes e resultados}
    
\end{frame}

\begin{frame}{Cronograma}
    
\end{frame}

\section{Considerações Finais}

\begin{frame}{Considerações Finais}
    
\end{frame}

{\setbeamercolor{palette primary}{fg=white, bg=mDarkTeal}
\begin{frame}[standout]
  Perguntas?
\end{frame}
}

\appendix


\begin{frame}[allowframebreaks]{References}

  \bibliography{demo}
  \bibliographystyle{abbrv}

\end{frame}

\end{document}
