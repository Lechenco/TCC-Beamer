\documentclass[10pt]{beamer}

\usetheme[progressbar=frametitle]{metropolis}
\usepackage{appendixnumberbeamer}

\usepackage{booktabs}
\usepackage[scale=2]{ccicons}

\usepackage{tabulary}
\usepackage[normalem]{ulem}

\usepackage{pgfplots}
\usepgfplotslibrary{dateplot}

\usepackage{xspace}
\newcommand{\themename}{\textbf{\textsc{metropolis}}\xspace}

\usepackage{multicol}

\title{Detecção de artefatos de arritmia utilizando Máquinas de Vetores de Suporte e Coeficientes de Energia Wavelet}
\subtitle{Proposta de TCC}
% \date{\today}
\date{2021}
\author{Gabriel Lechenco Vargas Pereira \\
Cristiano Marcos Agulhari}
\institute{Universidade Tecnológica Federal do Paraná - UTFPR}
% \titlegraphic{\hfill\includegraphics[height=1.5cm]{logo.pdf}} 

\begin{document}
\setbeamertemplate{frame footer}{Universidade Tecnológica Federal do Paraná}

\maketitle

% \begin{frame}{Sumário}
%   \setbeamertemplate{section in toc}[sections numbered]
%   \tableofcontents[hideallsubsections]
% \end{frame}

\section{Introdução}

\begin{frame}{Introdução}
    Estatísticas de doenças cardiovasculares: \cite{who_cardiovascular_2017}
    \begin{itemize}
        \item Foram responsáveis por um terço das mortes em 2016 
        \item Três quartos dessas ocorreram em países de baixa e média renda
        \item A principal causa das mortes são diagnóstico feito tardiamente 
    \end{itemize}
\end{frame}

\begin{frame} {Justificativa}
  O desenvolvimento de modelos computacionais com a capacidade de identificar 
  artefatos de arritmia pode auxiliar em um diagnóstico mais rápido de doenças 
  cardiovasculares.
\end{frame}

\begin{frame}{Objetivos Gerais}

  Este trabalho consiste em encontrar novas estratégias para 
  se diferenciar de forma automatizada trechos de eletrocardiograma em múltiplas 
  classes, a fim de identificar artefatos  de  arritmia  de  naturezas  diversas.

\end{frame}

\begin{frame}{Objetivos Específicos}

  % Dessa forma, serão utilizados:
  \begin{itemize}
    \item Identificar e recortar os trechos de sinal ECG relevantes
    \item Utilizar como vetor de características coeficientes de energia wavelet
    \item Treinar um modelo de Aprendizado de Máquina Supervisionado (SVM)
  \end{itemize}
\end{frame}

\section{Fundamentação Teórica}

\begin{frame}{Eletrocardiograma}
    \begin{figure}[]
      \centering
      \includegraphics[width=8cm]{images/pqrst.png}
      \caption{Ciclo PQRST \cite{faziludeen_ecg_2013}}
      \label{fig:pqrst}
    \end{figure}
\end{frame}

\begin{frame}{Arritmia}
  A falta de ritmo cardíaco tem ampla influência sobre a saúde do paciente.
    \begin{itemize}
      \item Deficiência no transporte e fornecimento de oxigênio.
      \item Podendo acarretar complicações em todo o corpo.
      \item Algumas capazes de levar ao óbito em poucos minutos.
    \end{itemize}
\end{frame}

\begin{frame}{Exemplo Eletrocardiograma}
  \begin{figure}
    \includegraphics[width=10.5cm]{images/NormalBeatSample.jpg}
    \caption{Trecho batimentos ritmo normal}
  \end{figure}

\end{frame}

\begin{frame}{Exemplo Eletrocardiograma}
  \begin{figure}
    \includegraphics[width=10.5cm]{images/AtrialFibrilaionSample.jpg}
    \caption{Trecho batimentos com fibrilação atrial}
  \end{figure}
\end{frame}

\begin{frame}{Exemplo Eletrocardiograma}
  \begin{figure}
    \includegraphics[width=10.5cm]{images/BigeminySample.jpg}
    \caption{Trecho batimentos com bigeminia ventricular}
  \end{figure}
\end{frame}

\begin{frame}{Exemplo Eletrocardiograma}
  \begin{figure}
    \includegraphics[width=10.5cm]{images/SinusBradicardiaSample.jpg}
    \caption{Trecho batimentos com bradicardia sinusal}
  \end{figure}
\end{frame}

\section*{Aprendizado de Máquina Supervisionado (SVM)}
\begin{frame}{Máquinas de Vetores de Suporte (SVM)}
  Algoritmo de classificação binária que busca encontrar o hiperplano 
  ótimo que seccione o hiperespaço onde os dados se encontram.

  \begin{equation}
    f(x) = \langle w, x \rangle + b = 0
    \nonumber
  \end{equation}
\end{frame}

\begin{frame}{Máquinas de Vetores de Suporte (SVM)}
  \begin{figure}[]
    \centering
    \includegraphics[width=10cm]{images/svmSample.png}
    \caption{Separação de dois planos por um hiperplano ótimo}
    \label{fig:svm}
  \end{figure}
\end{frame}

% \begin{frame}{Máquinas de Vetores de Suporte (SVM)}
%   Vantagens 
%   \begin{itemize}
%     \item Otimização de natureza convexa
%     \item Apresenta um unico mínimo global para problemas lineares
%     \item Consegue bons resultados com poucos exemplos
%   \end{itemize}

%   Desvantagens
%   \begin{itemize}
%     \item A princípio resolve apenas problemas lineares
%     \item Classificação binária
%   \end{itemize}
% \end{frame}

% \begin{frame}{SVM's e problemas não lineares}
%     \textbf{Teorema de Cover}
%     \begin{quote}
%       Dado um problema de classificação de padrões complexo, ao lançá-lo em 
%       um espaço com muitas dimensões é mais provável que este seja linearmente 
%       separável do que em um espaço com poucas dimensões, desde que o espaço não seja densamente preenchido.
%       \cite{haykin_neural_2010}
%     \end{quote}
% \end{frame}

\begin{frame} {SVM's e problemas não lineares}
  A adição de diferentes kernels possibilita uma maior flexibilidade do algoritmo de
  SVM com uma pequena modificação no problema de otimização.
  \begin{equation}
    f(x) = \langle w, \Phi(x) \rangle + b = 0
    \nonumber
  \end{equation}
\end{frame}

\begin{frame}{Problema XOR}
  \begin{figure}
    \centering
    \includegraphics[width=6cm]{images/svmXor.png}
    \caption{SVM utilizando o kernel gaussiano para o problema XOR}
  \end{figure}
\end{frame}

\begin{frame}{SVM's e problemas não binários}
  Técnicas para classificação não binária
  \begin{itemize}
    \item \textit{One Against All} (OAA)
    \item \textit{One Against One} (OAO)
    \item \textit{Directed Acyclic Graph SVM} (DAGSVM)
    \item \textit{Binary Tree of SVM} (BTS)
  \end{itemize}

\end{frame}

\section*{Extração de Características (Wavelet)}


\begin{frame} {Wavelets}
  \begin{figure}[]
    \centering
    \includegraphics[width=10cm]{images/waveAndWavelet.png}
    \caption{Onda com energia infinita e wavelet com energia concentrada \cite{burrus_introduction_1998}}
  \end{figure}
\end{frame}

\begin{frame} {Wavelets}
  Características das funções Wavelet:
  \begin{itemize}
    \item Infinitas funções wavelet disponíveis
    \item Localização em tempo-frequência (Escala e Translação)
    \item Multirresolução
  \end{itemize}
\end{frame}

\begin{frame}{Transformada de Wavelet Discreta}

  É possível reconstruir um sinal $f(t)$ partindo de uma 
  única função wavelet mãe $\psi$ e seus coeficientes de escala e translação $a_{j, k}$:

  \begin{equation}
    f(t) = \sum_{j, k} a_{j, k} \psi_{j, k} (t)
    \nonumber
  \end{equation}
  
\end{frame}

% \begin{frame}{Multirresolução}

%   A Transformada de Wavelet Discreta prioriza naturalmente uma melhor precisão em identificar 
%   as baixas frequências que compõem o sinal e uma melhor localização no tempo das altas 
%   frequências que causam fenômenos e perturbações momentâneas. \cite{strang_wavelets_1996}

% \end{frame}

\begin{frame}{\textit{Filter Bank} e \textit{Wavelet Packets}}

  \begin{figure}[]
    \centering
    \includegraphics[width=10cm]{images/filterBankPacket.png}
    \caption{Árvores de Decomposição Wavelet}
  \end{figure}
\end{frame}

% \section{Revisão de Literatura}

% \begin{frame}{Trabalhos Relacionados}
%   \begin{table}
%     \begin{tabulary}{\textwidth}{|C|C|}
%         \hline
%         Trabalho                               & Técnica   \\ \hline
%         Govindan, Deng e Power (1997)  & Wavelet + Redes Neurais                                                      \\ \hline
%         Zhao e Zhang (2005)        & Wavelet + SVM  + Modelagem Autorregressiva     \\ \hline
%         Mora e Amaya (2012)         & Entropia de Shannon + Complexidade de Lempel-Ziv + SVM-OAO assimétrica   \\ \hline
%         Rua et al. (2012)           & Energia Wavelet + Redes Neurais                                                                                      \\ \hline
%         Azariadi et al. (2016)          & Wavelet +   SVM            \\ \hline
%         Tuncer et al. (2019)      & Wavelet +  Localização de padrões  locais hexadecimais +  KNN  \\ \hline
%     \end{tabulary}
%   \end{table}
% \end{frame}

% \begin{frame}{Trabalhos Relacionados}
%   \begin{tabulary}{\textwidth}{|C|c|c|c|}
%     \hline
%     Trabalho                                                                          & Nº de classes & \begin{tabular}[c]{@{}l@{}}Nº de Exemplos \\no treinamento \end{tabular} & Acurácia       \\ \hline
%     Govindan, Deng e Power (1997)                                            & $4$                                                            &                      $10$                            & $77\% \pm 9\%$ \\ \hline
%     Zhao e Zhang (2005)                                                  & $6$                                                           &                 $7940$                                & $99,68\%$      \\ \hline
%     Mora e Amaya (2012)                                                   & $5$                                                            &                $637$                                 & $90,72\%$      \\ \hline
%     Rua et al. (2012)                                                     & $2$                                                           &               -                               & $99.46\%$      \\ \hline
%     Azariadi et al. (2016)                                                    & $2$                                                           &                      104581                         & $97\%$         \\ \hline
%     Tuncer et al. (2019)                                                & $17$                                                        &                        -                        & $95.0\%$       \\ \hline
%   \end{tabulary}
% \end{frame}

\section{Metodologia}

\begin{frame}{Metodologia}
  \begin{figure}[]
    \centering
    \includegraphics[height=6cm]{images/proposal.png}
    \caption{Descrição do Método que será utilizado}
  \end{figure}
\end{frame}

\begin{frame}{Bases de Dados}
  Bases de dados
  \begin{itemize}
    \item MIT-BIH Arrhythmia Database (mitdb)
    \begin{itemize}
      \item Trechos de arritmia
      \item Trechos saudáveis 
    \end{itemize}

    \item MIT-BIH Normal Sinus Rhythm Database (nsrdb)
    \begin{itemize}
      \item Trechos saudáveis
    \end{itemize}
  \end{itemize}
\end{frame}

\begin{frame}{Pré-processamento}
  Localizar e seccionar trechos selecionados:
  \begin{itemize}
    \item Ler anotações e comentários presentes nas bases de dados
    \item Localizar o início e término de eventos arrítmicos
    \item Seccionar trechos a cada 8 segundos
    \item Realizar reamostragem para $128 Hz$
  \end{itemize}
\end{frame}

% \begin{frame}{Extração de características}
%   Cada trecho de sinal será decomposto até o quarto nível, no final a folha 
%   mais a esquerda será decomposta por mais dois níveis. Será utilizada a 
%   função \textit{Daubechies} com suporte 3, por se adequar bem aos sinais cardíacos.
% \end{frame}

\begin{frame}{Estração de Características}
  \begin{figure}[]
    \centering
    \includegraphics[width=10.5cm]{images/waveletProposal.png}
    \caption{Decomposição Wavelet proposta}
  \end{figure}
\end{frame}

\begin{frame}{Aprendizado de Máquina}
    Máquinas de Vetores de Suporte (SVM)
    \begin{itemize}
      \item Aprendizado supervisionado
      \item Kernel Gaussiano
      \item \textit{K-fold} e \textit{Cross Validation}
      \item Comparação entre técnicas de classificação multiclasses
    \end{itemize}

  %   \begin{table}
  %     \centering
  %   \begin{tabular}{|c|c|}
  %     \hline
  %     Sigla   & Classe                        \\ \hline
  %     N       & Batimentos com ritmos normais \\ \hline
  %     AFIB    & Fibrilação Atrial             \\ \hline
  %     SBR     & Bradicardia Sinusal           \\ \hline
  %     B       & Bigeminia Ventricular         \\ \hline

  %   \end{tabular}
  % \end{table}
\end{frame}

\section{Resultados}

\begin{frame}{Precisão vs. \textit{Recall} }
  
    $$Precision = \frac{TP}{TP + FP}$$

    $$Recall = \frac{TP}{TP + FN}$$
  
\end{frame}

\begin{frame}{Matriz de Confusão}
  \begin{figure}[]
    \centering
    \includegraphics[width=8cm]{images/OAAConfusion.png}
    \caption{Matriz de Confusão para o modelo \textit{One Against All}}
  \end{figure}
\end{frame}

\begin{frame}{Matriz de Confusão}
  \begin{figure}[]
    \centering
    \includegraphics[width=8cm]{images/OAOConfusion.png}
    \caption{Matriz de Confusão para o modelo \textit{One Against One}}
  \end{figure}
\end{frame}

\begin{frame}{Matriz de Confusão}
  \begin{figure}[]
    \centering
    \includegraphics[width=8cm]{images/DAGSVMConfusion.png}
    \caption{Matriz de Confusão para o modelo DAGSVM}
  \end{figure}
\end{frame}

\begin{frame}{Matriz de Confusão}
  \begin{figure}[]
    \centering
    \includegraphics[width=8cm]{images/BTSConfusion.png}
    \caption{Matriz de Confusão para o modelo \textit{Binary Tree of SVM}}
  \end{figure}
\end{frame}

\begin{frame}{Precisão vs. \textit{Recall} }
  \begin{table}
    \centering
    \begin{tabular}{@{}llllll@{}}
      \toprule
      & AFIB   & B      & N      & SBR    & Average \\ \midrule
      One Against All      & 83.0\% & 80.4\% & 88.4\% & 87.9\% & 84.9\%  \\
      One Against One      & 74.6\% & 77.4\% & 82.8\% & 94.8\% & 82.4\%  \\
      Direct Acyclic Graph & 71.1\% & 77.4\% & 88.7\% & 94.8\% & 83.0\%  \\
      Binary Tree of SVM   & 78.3\% & 78.9\% & 88.9\% & 93.3\% & 84.9\%  \\ \bottomrule
    \end{tabular}
    \caption{Resumo da Precisão para cada um dos modelos testados.}
\end{table}

\end{frame}

\begin{frame}{Precisão vs. \textit{Recall} }
  \begin{table}
    \centering
    \begin{tabular}{@{}llllll@{}}
      \toprule
                         & AFIB   & B      & N      & SBR    & Average \\ 
    \midrule
    One Against All      & 73.3\% & 61.7\% & 63.3\% & 85.0\% & 70.8\%  \\ 
    One Against One      & 88.3\% & 68.3\% &  80.0\% & 91.7\% & 82.1\%  \\ 
    Direct Acyclic Graph & 90.0\% & 68.3\% & 78.3\% & 91.7\% & 82.1\%  \\ 
    Binary Tree of SVM   & 90.0\% & 75.0\% & 80.0\% & 93.3\% & 84.6\%  \\ 
    \bottomrule
    \end{tabular}
    \caption{Resumo da \textit{Recall} para cada um dos modelos testados.}
    
\end{table}

\end{frame}


% \begin{frame}{Cronograma}
%   \only<1>{
%     \begin{tabulary}{\textwidth}{|l|C|}
%       \hline
%       Data        & Atividade                                                                                    \\ \hline
%       14/Agosto   & Selecionar trechos relevantes dos sinais biológicos com base nas anotações do banco de dados \\ \hline
%       28/Agosto   & Realizar o janelamento e padronização destes trechos                                         \\ \hline
%       11/Setembro & Extrair Energias Wavelet                                                                     \\ \hline
%       02/Outubro  & Realizar Classificações                                                                      \\ \hline
%       23/Outubro  & Agrupar Resultados                                                                           \\ \hline
%       20/Novembro & Descrever Resultados e Conclusões finais                                                     \\ \hline
%       \end{tabulary}
%   }
%   \only<2>{
%     \begin{tabulary}{\textwidth}{|l|C|}
%       \hline
%       Data        & Atividade                                                                                    \\ \hline
%       14/Agosto   & \sout{Selecionar trechos relevantes dos sinais biológicos com base nas anotações do banco de dados} \\ \hline
%       28/Agosto   & \sout{Realizar o janelamento e padronização destes trechos                                        } \\ \hline
%       11/Setembro & \sout{ Extrair Energias Wavelet                                                                   } \\ \hline
%       02/Outubro  & \textbf{Realizar Classificações                                                                   } \\ \hline
%       23/Outubro  & Agrupar Resultados                                                                           \\ \hline
%       20/Novembro & Descrever Resultados e Conclusões finais                                                     \\ \hline
%       \end{tabulary}
%   }
% \end{frame}

\section{Conclusão}

\begin{frame}{Conclusão}
  Trabalhos Futuros
  \begin{itemize}
    \item Combinar os coeficientes de energia Wavelet com novas características
    \item Investigar classificações equivocadas
  \end{itemize}

  Considerações Finais
  \begin{itemize}
    \item A \textit{Binary Tree of SVM} se destacou das demais
    \item A solução proposta alcançou resultados favoráveis
  \end{itemize}
\end{frame}

 {\setbeamercolor{palette primary}{fg=white, bg=mDarkTeal}
 \begin{frame}[standout]
   Perguntas?
 \end{frame}
}

\appendix


\begin{frame}[allowframebreaks]{References}

  \bibliography{demo}
  \bibliographystyle{abbrv}

\end{frame}



\begin{frame}
  \only<1>{\textit{Directed Acyclic Graph SVM} (DAGSVM)
  \begin{figure}[]
    \centering
    \includegraphics[width=6cm]{images/dagSVM.png}
  \end{figure}
}
\only<2>{\textit{Binary Tree of SVM} (BTS)
  \begin{figure}[]
    \centering
    \includegraphics[width=7cm]{images/btsSample.png}
  \end{figure}
}
\end{frame}



\end{document}
